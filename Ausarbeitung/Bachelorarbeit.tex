\documentclass[11pt,a4paper,twoside,openright]{scrbook}
\usepackage{clba}

% Per Kapitel Nummerierung von Graphiken und Tabellen
\usepackage{chngcntr}
\counterwithin{figure}{chapter}
\counterwithin{table}{chapter}


% Hier die eigenen Daten eintragen
\global\fach{Computerlinguistik}
\global\arbeit{Bachelorarbeit}
\global\titel{Character-Level Recurrent Neural Networks}
\global\bearbeiter{Leonie Weißweiler}
\global\betreuer{Prof. Anna Korhonen}
\global\pruefer{Prof. Alexander Fraser}
\global\universitaet{Ludwig- Maximilians- Universität München}
\global\fakultaet{Fakultät für Sprach- und Literaturwissenschaften}
\global\department{Department 2}

\global\abgabetermin{28. Mai 2018}
\global\bearbeitungszeit{26. März - 28. Mai 2012}
\global\ort{München}


\begin{document}

% Deckblatt
\deckblatt

\pagestyle{scrheadings}
\pagenumbering{gobble}

% Erklärung fürs Prüfungsamt
\erklaerung

% Zusammenfassung
\addchap{Abstract}
\thispagestyle{scrplain}
\noindent
Das ist ein Abstract (hier max. 250 Wörter)

% Inhaltsverzeichnis
\pagenumbering{Roman}

\tableofcontents

% Text mit arabischer Nummerierung
\pagenumbering{arabic}

\chapter{Introduction}
\cite{irbook}
\section{Ein Abschnitt}

Mein Name ist Hase und ich weiß von nichts. Das ist ein Testtext. Mein Name ist
Igel und ich weiß auch von nichts.

\subsection{Ein Unterabschnitt}
Blabla. Hier ein Unterabschnitt.

\subsubsection{Ein Unterunterabschnitt}
\label{sec:a}
Blabla. Hier ein Unterunterabschnitt.

\subsubsection{Noch ein Unterunterabschnitt}
\label{sec:b}

Wer \ref{sec:a} sagt, muss auch \ref{sec:b} sagen.

\subsection{Noch ein Unterabschnitt}

Das ist ein gewöhnlicher Absatz.

\paragraph{Ein Absatz mit Titel}
Paragraphen gibts auch.

\subparagraph{Ein Unterabsatz mit Titel}
Und dann auch noch Unterparagraphen.

\subsection*{Ein nicht nummerierter Unterabschnitt}
Dieser Unterabschnitt erscheint nicht im Inhaltsverzeichnis.
\newpage

\section{Beispiele}
Blabla.
\newpage

\section{Mehr Beispiele}
Blabla.
\newpage

\chapter{Introduction}
\cite{irbook}

\chapter{Background}
\section{Recurrent Neural Networks}
\section{Long-Short Time Memory}
\section{Neural Architecture Search}
\section{Gated Recurrent Units}
\section{Heap's Law}

\chapter{Related Work}

\chapter{Datasets and Evaluation Setup}
\section{The Wikipedia Corpus}
\section{Preprocessing}
\section{Evaluation setup}
\section{Evaluation of RNN output quality}
\section{Evaluation of RNN productiveness}

\chapter{Interesting Evaluation Observations}

\chapter{Conclusion and Future Work}
\section{Conclusion}
\section{The Future of Character-Level RNNs as language models}
\section{Character-Level RNNs for the sparsity problem}

%Beispielliteratur
\bibliographystyle{apalike}
\bibliography{Bachelorarbeit}
\newpage

% Abbildungsverzeichnis (kann auch nach dem Inhaltsverzeichnis kommen)
\listoffigures
\newpage

% Tabellenverzeichnis (kann auch nach dem Inhaltsverzeichnis kommen)
\listoftables
\newpage

\addchap{CD contents}

\end{document}
